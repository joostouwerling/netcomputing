\section{Introduction}
Professional sports is a matter of details. The difference between winning or losing can be extremely small. In order to maximise sport performance, data is necessary. A coach or manager wishes to have a lot of data how a athlete performs during activities. Data what could be collected can be speed, covered distance, positions and other location-based data.

A good abstraction of how a sports(wo)man performs is illustrated in a heat-map. A heat-map is a graphical representation where individual values contained in a matrix are represented as colors. A heat-map provides a great abstraction of where an athlete was during a game. Positions during matches can be highlighted and intensity can be increased according to their attendance on a position.

Our idea is to design a program which creates a heat-map during a match, in order to afterwards analyse these heatmaps. A player has the app on his/her phone which sends his locations to a program. A program receives this locations and extracts heatmaps from the locations. A coach can access these heatmaps and analyse the game of his players.